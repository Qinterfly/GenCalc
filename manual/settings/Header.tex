
% -- Настройки класса документа -------------------------------------------

\documentclass[12pt, a4paper, oneside, final]{book} % Класс документа
\usepackage{cmap} 								    % Поиск и копирование текста
\usepackage[T2A]{fontenc} 							% Специалньые символы
\usepackage[utf8]{inputenc} 						% Кодировка
\usepackage[english, russian]{babel} 				% Используемые языки

% -- Настройка шрифтов ----------------------------------------------------

\usepackage[scaled = 0.96, sups]{XCharter} 			 % Bitstream Charter ("< \ "> -- кавычки)
\usepackage[scaled = 1.04, varqu, varl]{inconsolata} 
\usepackage[type1]{cabin} 							 % Шрифт для текста
\usepackage[vvarbb, scaled = 1.12]{newtxmath} 		 % Шрифт для формул (libertine)
\usepackage[cal = boondoxo]{mathalfa} 				 % Гарнитура шрифта для формул
 
% -- Загрузка библиотек ---------------------------------------------------

%\usepackage{amssymb}
\usepackage{amsmath, amsfonts, latexsym} 				% Математическая символика
\usepackage{geometry} 					 				% Поля
\usepackage{setspace}                    				% Коррекция интервалов
\usepackage{indentfirst}                 				% Начало главы с красной строки
% \usepackage[none]{hyphenat}            				% Отключение переноса слов
\usepackage{graphicx}                    				% Графика
\usepackage{titlesec}                    				% Редактирование section и chapter
\usepackage{fancyhdr, fancybox}          				% Редактирование колонтитулов
\usepackage{listings}                    				% Оформление листинга кода с подсветкой
\usepackage{color}                       				% Цвета
\usepackage{hyperref}                    				% Гиперссылки в PDF
\usepackage{nicefrac}                    				% Симпатичные дроби
\usepackage{float}                       				% Настройка плавающего окружения
\usepackage{rotating}                    				% Поворот плавающего окружения
\usepackage{ulem}                        				% Настраиваемое подчеркивание текста
\usepackage{titlesec}                    				% Настройка отображения заголовков
\usepackage{enumitem}                    				% Настройка форматирования нумерованных списков
\usepackage{algorithm}                   				% Использование псевдокодов
\usepackage{algpseudocode}                              % Поддержка русского языка в псевдокодах
\usepackage{tikz}                        				% Создание векторной графики
\usepackage{xfp}                         			    % Поддержка вычислений в коде
\usepackage{framed}                     			    % Выделение блоком текста
\usepackage{pgfplots}                    				% Построение графиков
\usepackage{stanli}                      				% Пакет Tikz для отрисовки конструкций
\usepackage{tabularray}                  				% Таблицы с гибким форматированием
\usepackage{cite}										% Группировка источников при цитировании
\usepackage[labelfont = {it}, textfont = {it}]{caption} % Настройка подписей
\usepackage[nottoc, notlof, notlot]{tocbibind} 			% Отображение списка источников в оглавлении
\graphicspath{{images//}} 								% Путь для картинок (//==рекурсивный поиск директорий)

% -- Опции форматирования ---------------------------------------------------	

\geometry{left = 3cm, right = 1.5cm, top = 2cm, bottom = 2cm} % Настройки полей
\frenchspacing 												  % Одинаковые пробелы
\renewcommand{\baselinestretch}{1.3}					      % Межстрочный интервал
\parindent = 1.25cm 										  % Отступ абзацный
\hyphenpenalty = 5000 										  % Пороговое значение штрафа для переноса слова
% \hbadness = 10000 										  % Отключение предупреждения о проблемных блоках (при hypehenat==on)
\sloppy 													  % Контроль переполнения блоков

% -- Работа с таблицами -----------------------------------------------------

% Tabular и Longtable
\DeclareCaptionLabelFormat{gost-table}{\hfill #1~#2}    % Указатель
\DeclareCaptionTextFormat{gost-table}{\centering #1 \\} % Текст
\captionsetup[table]{textformat = gost-table, labelformat = gost-table, singlelinecheck = off, labelsep = newline} % Задание формата

% Tabularrray
	% Стиль
\NewTblrTheme{regularTable}{
	\SetTblrStyle{caption-tag}{font=\itshape}
	\SetTblrStyle{caption-text}{font=\itshape}
}
\UseTblrLibrary{varwidth}
	% Настройки заголовка первого блока таблиц
\DefTblrTemplate{caption-sep}{default}{}
\DefTblrTemplate{caption}{default}{
\par\hfill
	\UseTblrTemplate{caption-tag}{default}
	\UseTblrTemplate{caption-sep}{default}
	\newline\centering
    \UseTblrTemplate{caption-text}{default}	
   	\par
}
	% Настройки перенесенного заголовка
\DefTblrTemplate{contfoot-text}{default}{}
\DefTblrTemplate{conthead-text}{default}{(продолжение)}
\DefTblrTemplate{capcont}{default}{
\par\hfill
	\UseTblrTemplate{caption-tag}{default}
	\UseTblrTemplate{caption-sep}{default}
	\newline\centering
    \UseTblrTemplate{caption-text}{default}	
    \UseTblrTemplate{conthead-text}{default}
\par
}

% -- Нумерация страниц ------------------------------------------------------

% Смена заголовков глав и секций
\pagestyle{fancy} % Выбор системного стиля
\renewcommand{\chaptermark}[1]{\markboth{\itshape\thechapter.\ #1}{}} % Главы
\renewcommand{\sectionmark}[1]{\markright{\itshape\thesection.\ #1}}  % Секции

% Базовый стиль (начало глав)
\fancypagestyle{plain}{
	\fancyhf{} 						   % Очистка всех хедеров и футеров
	\fancyfoot[C]{-- \thepage\ --} 	   % Нумерация страницы
	\renewcommand{\headrulewidth}{0pt} % Толщина разделительной полосы сверху
	\renewcommand{\footrulewidth}{0pt} % Толщина разделительной полосы снизу
}
 % Основной стиль
\fancypagestyle{main}{
	\fancyhf{}											   % Очистка всех хедеров и футеров
	% Положение глав
	\fancyhead[L]{\ifodd\value{page} \leftmark \fi} 	   % Нечетные страницы
	\fancyhead[R]{\ifodd\value{page} \else \rightmark \fi} % Четные страницы
	\fancyfoot[C]{-- \thepage\ --} 						   % Нумерация страницы
	\renewcommand{\headrulewidth}{0.2pt} 				   % Толщина разделительной полосы сверху
	\renewcommand{\footrulewidth}{0pt} 					   % Толщина разделительной полосы снизу
}

% Задание дополнительных стилей
	% Стиль нумерации без хедера
\fancypagestyle{onlyNum}{ 
	\fancyhf{} 						   % Очистка всех хедеров и футеров
	\fancyfoot[C]{-- \thepage\ --}     % Нумерация страницы
	\renewcommand{\headrulewidth}{0pt} % Толщина разделительной полосы сверху
	\renewcommand{\footrulewidth}{0pt} % Толщина разделительной полосы снизу
}
	% Стиль нумерации без глав
\fancypagestyle{onlyRightMark}{ 
	\fancyhf{} 														% Очистка всех хедеров и футеров
	\fancyfoot[C]{-- \thepage\ --} 								    % Нумерация страницы
	\fancyhead[R]{\ifodd\value{page} \else \bfseries\rightmark \fi} % Четные страницы
	\fancyfoot[C]{\thepage} 										% Нумерация страницы
}

%  -- Рубрикация -------------------------------------------------------------

\makeatletter % Используем символ @ как букву
% Запрет переносов в названиях секций
\renewcommand{\section}{\@startsection{section}{1}{0pt}{1ex}{1ex}{\centering\hyphenpenalty=10000\normalfont\bfseries}}

% Смена заголовка для команды \chapter и запрет переноса слов
\titleformat{\chapter}{\centering\hyphenpenalty=10000\normalfont\large\bfseries}{\thechapter. }{0pt}{\large}
\renewcommand\@biblabel[1]{#1.} % Точка перед источниками литературы
\makeatother
\titlespacing*{\chapter}{0pt}{-50pt}{10pt} % Вертикальные отступы от названия главы
\titleformat*{\subsection}{\normalfont\bfseries} % Подраздел

% Кириллический алфавит
\makeatletter
\renewcommand*{\@alph}[1]{% 
  \ifcase#1\or а\or б\or в\or г\or
    д\or е\or ё\or ж\or з\or и\or й\or
    к\or л\or м\or н\or о\or п\or р\or с\or т\or
    у\or ф\or х\or ц\or ч\or
    ш\or щ\or ъ\else\@ctrerr\fi
}
\renewcommand*{\@Alph}[1]{% 
  \ifcase#1\or А\or Б\or В\or Г\or
    Д\or E\or Ё\or Ж\or З\or И\or Й\or
    К\or Л\or М\or Н\or О\or П\or Р\or С\or Т\or
    У\or Ф\or Х\or Ц\or Ч\or
    Ш\or Щ\or Ъ\else\@ctrerr\fi
}
\makeatother

%  -- Листинг кода -------------------------------------------------------------

% Смена названия блока псевдокода
\makeatletter
\renewcommand*{\ALG@name}{Алгоритм}
\makeatother

% Определение именнованных цветов
\definecolor{mygreen}{RGB}{28, 172, 0} 
\definecolor{mylilas}{RGB}{170, 55, 241} 
	
%  Общие настройки
\lstset{
	basicstyle = \small\normalfont,    % Размер и начертание шрифта для подсветки кода
	numbers = left,                    % Где поставить нумерацию строк (слева\справа)
	numberstyle = \tiny,               % Размер шрифта для номеров строк
	stepnumber = 1,                    % Размер шага между двумя номерами строк
	numbersep = 5pt,                   % Как далеко отстоят номера строк от подсвечиваемого кода
	% backgroundcolor = \color{white}, % Цвет фона подсветки
	showspaces = false,                % Показывать или нет пробелы специальными отступами
	showstringspaces = false,          % Показывать или нет пробелы в строках
	showtabs = false,                  % Показывать или нет табуляцию в строках
	frame = tb,                 	   % Рисовать рамку вокруг кода
	tabsize = 2,                       % Размер табуляции по умолчанию равен 2 пробелам
	captionpos = t,                    % Позиция заголовка вверху [t] или внизу [b] 
	breaklines = true,                 % Автоматически переносить строки (да\нет)
	breakatwhitespace = false,         % Переносить строки только если есть пробел
	texcl = true, 			           % Поддержка русских символов (LaTeX коммент.)
}

% Стили языков
	% Matlab 
\lstdefinestyle{Matlab}{ 
	language = Matlab,                                        % Выбор языка для подсветки
	morekeywords = {matlab2tikz},							  % Подгрузка ключевых слов
	keywordstyle = \color{blue},                              % Подсветка ключевых слов
	morekeywords = [2]{1}, keywordstyle = [2]{\color{black}}, % Стиль ключевых слов
	identifierstyle = \color{black},						  % Фильтрация стиля
	stringstyle = \color{mylilas},						      % Цвет строк
	commentstyle = \color{mygreen},						      % Цвет комментариев
	showstringspaces = false, 							      % Пробельный символ
	numbers = left,									          % Нумерация слева
	numberstyle = {\tiny \color{black}},					  % Шрифт нумерации
	numbersep = 9pt, 									      % Отступ нумерации
	emph = [1]{for,end,break}, emphstyle = [1]\color{red}, 	  % Акцент на словах
}	

	% Tree
\lstdefinestyle{tree}{
	language = Matlab, 
	stringstyle = \color{mylilas},	% Цвет строк
	commentstyle = \color{mygreen},	% Цвет комментариев
	showstringspaces = false, 		% Пробельный символ
	numbers = none,					% Нумерация слева	
    literate =						% Замена символов дерева утилиты tree
    {├}{{\smash{\raisebox{-1ex}{\rule{1pt}{\baselineskip}}}\raisebox{0.5ex}{\rule{1ex}{1pt}}}}1 
    {─}{{\raisebox{0.5ex}{\rule{1.5ex}{1pt}}}}1 
    {└}{{\smash{\raisebox{0.5ex}{\rule{1pt}{\dimexpr\baselineskip-1.5ex}}}\raisebox{0.5ex}{\rule{1ex}{1pt}}}}1,
    emph = [1]{Signals, Results, LICENSE, Geometry, Distortion field, Export_fig},emphstyle=[1]\color{red}, 	  % Акцент на словах
}

	% Fortran
\lstdefinestyle{Fortran}{ 
	language = Fortran,                                       % Выбор языка для подсветки
	breaklines = true,									      % Пропуск линий
	keywordstyle = \color{blue},                              % Подсветка ключевых слов
	morekeywords = [2]{1}, keywordstyle = [2]{\color{black}}, % Стиль ключевых слов
	identifierstyle = \color{black},						  % Фильтрация стиля
	stringstyle = \color{mylilas},						      % Цвет строк
	commentstyle = \color{mygreen},						      % Цвет комментариев
	showstringspaces = false, 							      % Пробельный символ
	numbers = left,									          % Нумерация слева
	numberstyle = {\tiny \color{black}},					  % Шрифт нумерации
	numbersep= 9pt, 									      % Отступ нумерации
	emph = [1]{for, end, break}, emphstyle = [1]\color{red},  % Акцент на словах
}	

% -- Работа с графикой --------------------------------------------------------

\usetikzlibrary{shadows, arrows}          % Использование теней и стрелок
\usetikzlibrary{positioning}              % Позиционирование
\usetikzlibrary{patterns}                 % Паттерны заливки
\usetikzlibrary{decorations.pathmorphing} % Декорации
\usetikzlibrary{calc}                     % Расчетная библиотека
\pgfplotsset{compat = newest}             % Версия библиотеки для отрисовки графиков
\usetikzlibrary{trees}   				  % Деревья

% Чертежные виды
\tikzstyle{isometric} = [x = {(0.710cm, -0.410cm)}, y = {(0cm, 0.820cm)}, z = {(-0.710cm,-0.410cm)}]
\tikzstyle{dimetric}  = [x = {(0.935cm, -0.118cm)}, y = {(0cm, 0.943cm)}, z = {(-0.354cm, -0.312cm)}]
\tikzstyle{dimetric2} = [x = {(0.935cm, -0.118cm)}, z = {(0cm, 0.943cm)}, y = {(+0.354cm, +0.312cm)}]
\tikzstyle{trimetric} = [x = {(0.926cm, -0.207cm)},y = {(0cm, 0.837cm)}, z = {(-0.378cm, -0.507cm)}]

% Стили
	% Оси и размеры
\tikzstyle{dim<->} = [thick, latex-latex]              % Двусторонний размер
\tikzstyle{dim->} = [-latex] 						   % Односторонний размер
\tikzstyle{axis} = [thick, -latex', color = black]     % Линия оси координат
\tikzstyle{symLine} = [thick, dash dot, color = black] % Линия симметрии
	% Штриховка
\tikzstyle{hatching1} = [pattern = north east lines]   % Нельзя переопределять hatching из stanli (!)
	% Вектора
\tikzstyle{vector} = [very thick, -latex]

%  -- Дополнительные математические операторы ---------------------------------

\DeclareMathOperator*{\argmax}{arg\,max}
\DeclareMathOperator*{\argmin}{arg\,min}

%  -- Навигация ---------------------------------------------------------------

% Гиперссылки
\hypersetup{ 
    colorlinks = true, 	 % Включить цветные ссылки (print == false)
    linkcolor = blue,	 % Цвет ссылки на объект
    filecolor = magenta, % Цвет ссылки на файл    
    urlcolor = cyan,	 % Цвет ссылки на сайт
    citecolor = green 	 % Цвет ссылки на цитату
}

% -- Завершение статической части преамбулы ------------------------------------

\csname endofdump \endcsname

% ------------------------------------------------------------------------------