
% -- Переименование рубрикации -------------------------------------------------

\renewcommand{\contentsname}{Содержание} 				  % Оглавление -> Содержание
\renewcommand{\bibname}{Список использованных источников} % Литература -> Список использованных источников

% -- Главы и секции без нумерации ---------------------------------------------

\newcommand{\AddNoTocChap}[2]{ % Использование глав без нумерации
\def\temp{#2}\ifx\temp\empty % Проверка типа главы
	\chapter*{#1}  % Введение, заключение...
  	\addcontentsline{toc}{chapter}{#1} \fancyhead{} % Сброс заголовка хедера
\else
  	\chapter*{#1 \\ #2}  % Приложение
  	\addcontentsline{toc}{chapter}{#1. #2} \fancyhead{} % Сброс заголовка хедера
\fi
\fancyhead[L]{\ifodd\value{page} \bfseries \MakeUppercase{#1} \fi} % Нечетные страницы
\fancyhead[R]{\ifodd\value{page} \else \bfseries \MakeUppercase{#1} \fi} % Четные страницы
}

% -- Создание математических формул -------------------------------------------

% Частная производная n-го порядка по одному аргументу
\newcommand{\partDer}[3]{ 
\ifthenelse{ \equal{#3}{1} \OR \equal{#3}{} }
	{ \frac{\partial #1}{\partial #2} }
	{ \frac{\partial^{#3} #1}{\partial #2^{#3}} }
}
% Верхнее подчеркивание
\newcommand{\ol}[1]{ 
 	\overline{#1}
}
\setlength{\abovedisplayskip}{1ex} % Вертикальный отступ перед формульным окружением
\setlength{\belowdisplayskip}{1ex} % Вертикальный отступ после формульного окружения

% -- Сокращения ---------------------------------------------------------------

% Формулы
\newcommand{\nfrac}{\nicefrac}        % Однострочная дробь
\newcommand{\rarr}{$\rightarrow$ }    % Стрелка вправо
\newcommand{\matr}[1]{\mathbf{#1}}    % Жирные матричные символы
\newcommand{\vphi}{\varphi}        	  % Phi
\newcommand{\R}{\mathbb{R}}           % Множество
\newcommand{\imag}{\operatorname{Im}} % Мнимая часть числа
\newcommand{\real}{\operatorname{Re}} % Действительная часть числа

% Рисунки
\newcommand{\figrefb}[1]{рис.~\ref{#1}} % Без скобок
\newcommand{\figref}[1]{(\figrefb{#1})} % Со скобками
% Таблицы
\newcommand{\tabrefb}[1]{табл.~\ref{#1}} % Без скобок
\newcommand{\tabref}[1]{(\tabrefb{#1})}  % Со скобками
% Листинги
\newcommand{\listref}[1]{(лист.~\ref{#1}, c.~\pageref{#1})} 

% -- Обновление fancy ---------------------------------------------------------

\newcommand{\updateFancy}{ 
	\pagestyle{main}
}

% -- Установки по умолчанию ---------------------------------------------------

\updateFancy % Стиль нумерации

% -----------------------------------------------------------------------------